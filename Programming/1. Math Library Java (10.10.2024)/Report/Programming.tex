\documentclass[12pt]{article}

\usepackage[a4paper, left=3cm, right=1.5cm, top=2cm, bottom=2cm]{geometry}

\setlength{\parindent}{1.25cm}

\usepackage[utf8]{inputenc}
\usepackage{mathtools}
\usepackage[russian, english]{babel}
\usepackage{multirow}
\usepackage{graphicx}
\usepackage{listings}
\usepackage{xcolor}
\usepackage{float}
\graphicspath{{/home/oriptal/Pictures/Screenshots}}

\addto\captionsenglish{\renewcommand{\contentsname}{\huge Table of Contents}}
\addto\captionsenglish{\renewcommand{\figurename}{Fig.}}

\begin{document}
	\begin{center}
		\thispagestyle{empty}
		\LARGE
		\textbf{NRU ITMO}\\
		SEaCT\\
		Programming
		
		\vspace{7cm}
		
		\huge
		
		\textbf{Laboratory Work №1}\\
		\vspace{2cm}
		\Large
		Variant 311607 
		
		\LARGE
		\vspace{5cm}
		\vbox{
			\hfill
			\vbox{
				\hbox{By: Loskutov P.\,A.}
				\hbox{To: Naumova N.\,A.}
			}
		} 
		
		\vspace{2.5cm}
		Saint Petersburg, 2024
		\newpage
		\tableofcontents
	\end{center}
	\newpage
	\section{\LARGE Problems}
	\Large
	\begin{enumerate}
		\item Create \(w\) - int[17], \(x\) - double[17], \(w_1\) - double[17][17].
		\item Fill arrays this way:
		\begin{itemize}
		\item \(\forall i \in [0..16],\,w[i] = 17 - i\)
		\item \(\forall i \in [0..16],\,x[i] = rnd(-13.0, 4.0)\)
			\item \(\forall i: w[i] = 3 \rightarrow \forall j, w_1[i][j] = \arcsin(\frac{1}{e^{3(\tan^2(x)+1)^{\tan(x^{2x})})}})\)
			\item \(\forall i: w[i] \in \{1, 9, 10, 12, 13, 14, 15, 16\} \rightarrow\\\rightarrow \forall j, w_1[i][j] = (\arcsin(\frac{x-4.5}{17})^{(\frac{\frac{2}{3}+\sqrt{3}x}{0.5})})^{\frac{\sqrt[3]{\arctan(\frac{x-4.5}{17})}}{2}}\)
			\item else: \(\forall j,w_1[i][j] = ((\frac{(\arcsin(\frac{x-4.5}{17}))^{3}}{1}/4)^{(\frac{\sqrt[3]{x}}{2})^3})^{\frac{1-\frac{\frac{1}{2}-\sin(e^{x})}{4}/(\frac{x}{3-x})^2}{\arcsin(0.4\times e^{-|x|})}}
		\)
		\end{itemize}
	\end{enumerate}
	\newpage
	\section{\LARGE Code}
	\begin{figure}[H]
		\setlength{\fboxsep}{0pt}
		\setlength{\fboxrule}{0pt}
		\begin{center}
			\lstset{language=Java, showspaces=false,
				showstringspaces=false, tabsize=2, breaklines=true, keywordstyle=\color{magenta}}
			\begin{lstlisting}
				public class Main {
					static final int n = 17;
					static int[] w;
					static double[] x;
					static double[][] w1; 
					
					static void print(double[][] a) {
						for (int i = 0; i < n; i++) {
							for (int j = 0; j < n; j++) {
								System.out.printf("%7.5f", a[i][j]);
								System.out.print(' ');
							}
							System.out.println();
						}
					}
					
					public static void main(String[] args) {   
						exFirst();
						exSecond();
						exThird();
						print(w1);
					}
					
					static void exFirst() {
						w = new int[n];
						for (int i = 0; i < n; i++) {
							w[i] = 17 - i;
						}
					} 
					
					static void exSecond() {
						x = new double[n];
						for (int i = 0; i < n; i++) {
							x[i] = (Math.random() * n) - 13;
						}
					}
					
					static void exThird() {
						w1 = new double[n][n];
						for (int i = 0; i < n; i++) {
							for (int j = 0; j < n; j++) {
								w1[i][j] = count(w1[i][j], i, j);
							}
						}
					}
				}
			\end{lstlisting}
		\end{center}
	\end{figure}
	\begin{figure}[H]
		\setlength{\fboxsep}{0pt}
		\setlength{\fboxrule}{0pt}
		\begin{center}
			\lstset{language=Java, showspaces=false,
				showstringspaces=false, tabsize=2, breaklines=true, keywordstyle=\color{magenta}}
			\begin{lstlisting}
					static double count(double e_w1, int i, int j) {
						e_w1 = switch (w[i]) {
							case 3 -> firstFunc(x[j]);
							case 1, 9, 10, 12, 13, 14, 15, 16 -> secFunc(x[j]);
							default -> thirdFunc(x[j]);
						};
						return e_w1;
					}
					
					static double firstFunc(double x) {
						double upper_power = Math.tan(Math.pow(x, 2*x));
						double lower_power = Math.pow((3*(Math.pow(Math.tan(x),2)+1)), upper_power);
						double denom = Math.exp(lower_power);
						double ans = Math.asin(1.0/denom);
						return ans;
					}
					
					static double secFunc(double x) {
						double arg = (x - 4.5) / 17.0;
						double upper_power = Math.pow(Math.atan(arg), 1.0/3.0) / 2.0;
						double lower_power = ((2.0/3.0) + Math.pow(x, 1.0/3.0)) / 0.5;
						double inside = Math.pow(Math.asin(arg), lower_power);
						double ans = Math.pow(inside, upper_power);
						return ans;
					} 
					
					static double thirdFunc(double x) {
						double strange_lower_pow = x / 8.0;
						double arg = Math.pow(Math.pow(Math.asin((x - 4.5) / 17.0), 3.0) / 4.0, strange_lower_pow);
						double denom = Math.asin(0.4*Math.exp(-Math.abs(x)));
						double nom_denom = Math.pow(x/(3.0 - x), 2);
						double nom = 1 - ((0.5 - Math.sin(Math.exp(x)))/4.0)/nom_denom;
						double arg_pow = nom / denom;
						double ans = Math.pow((arg), arg_pow);
						return ans;
					} 
				}
			\end{lstlisting}
		\end{center}
	\end{figure}
\end{document}

